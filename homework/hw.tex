\documentclass[lang=cn,11pt]{elegantbook}
\usepackage[utf8]{inputenc}
\usepackage[UTF8]{ctex}
\usepackage{amsmath}%
\usepackage{amssymb}%

\title{597 Homeworks}

\begin{document}
\frontmatter
\tableofcontents
\mainmatter



\chapter*{preliminary}

\noindent (Not graded.) Read Sections 0.1-0.3 and 0.5-0.6 in Folland’s book. Note: I expect you to have seen much but not necessarily all of this material in earlier courses. It is not necessary to know everything by heart right now. However, in order to succeed in the class, you need to be able to read mathematical material at this level of abstraction and (lack of) detail.

\section{Approaching 597}
Let $A$ be an infinite (not necessarily countable) set, and $f : A \to \mathbb{R}$ a function. Suppose that for every integer $N \geq 1$ there exist finite subsets $A^+_N \subset A$ and $A^-_N \subset A$ such that:
\begin{itemize}
    \item (i) $\lvert f(\alpha) \rvert \leq N^{-1}$ for all $\alpha \in A \setminus (A^+_N \cup A^-_N)$;
    \item (ii) $\sum_{\alpha \in A^+_N} f(\alpha) \geq N$;
    \item (iii) $\sum_{\alpha \in A^-_N} f(\alpha) \leq -N$.
\end{itemize}
Prove that for any $N \geq 1$, there exists a finite subset $B_N \subset A$ such that
\[
\left\lvert 597 - \sum_{\alpha \in B_N} f(\alpha) \right\rvert \leq \frac{1}{N}.
\]

\section{Limsup and Liminf}
Let $X$ be a nonempty set, and $A, B$ subsets of $X$. Define a sequence $(E_n)_{n=1}^\infty$ of subsets of $X$ by
\[
E_n = 
\begin{cases} 
A & \text{if } n \text{ is a prime number,} \\
B & \text{otherwise.}
\end{cases}
\]
Characterize the sets $\limsup E_n$ and $\liminf E_n$ (see §0.1 in Folland for notation).

\section{Polynomial Convergence}
Let $f : \mathbb{Z}_{\geq 0} \times \mathbb{Z}_{\geq 0} \to \mathbb{R}$ be a function with the property that for every polynomial
\[
p(x) = x^d + a_1 x^{d-1} + \cdots + a_d
\]
with integer coefficients, we have that
\[
\lim_{n \to \infty} f(n, p(n)) = \lim_{n \to \infty} f(p(n), n) = 0.
\]
Does it follow that $f(m,n) \to 0$ as $m, n \to \infty$? In other words, given $\epsilon > 0$, does there exist $N \geq 0$ such that $\lvert f(m,n) \rvert < \epsilon$ whenever $\lvert m \rvert, \lvert n \rvert \geq N$? Give a proof or a counterexample.




\chapter{on $\sigma$-algebra}

\section{Borel vs Open}
Let $X$ be a metric space such that every subset of $X$ is Borel measurable. Does it follow that every subset of $X$ is open? Give a proof or a counterexample.

\section{Restriction of a $\sigma$-algebra to a Subset}
Let $X$ be a set, and $Y \subset X$ a subset.

\begin{itemize}
    \item[(a)] Given a $\sigma$-algebra $\mathcal{A}$ on $X$, prove that
    \[
    \mathcal{A}|_Y := \{E \cap Y \mid E \in \mathcal{A}\}
    \]
    is a $\sigma$-algebra on $Y$.
    \item[(b)] Given a $\sigma$-algebra $\mathcal{B}$ on $Y$, prove that there exists a $\sigma$-algebra $\mathcal{A}$ on $X$ such that $\mathcal{A}|_Y = \mathcal{B}$.
    \item[(c)] Is the $\sigma$-algebra $\mathcal{B}$ in (b) unique? Give a proof or a counterexample.
\end{itemize}

\section{Invariance Properties of the Borel $\sigma$-algebra on $\mathbb{R}^n$}
\begin{itemize}
    \item[(a)] Prove that $\mathcal{B}(\mathbb{R}^n)$ is translation invariant, i.e., if $A \subset \mathbb{R}^n$ is a Borel measurable set, then
    \[
    t + A := \{t + x \mid x \in A\}
    \]
    is a Borel measurable set for every $t \in \mathbb{R}^n$. (Hint: For any fixed $t$, show that $A = \{B \subset \mathbb{R}^n : t + B \in \mathcal{B}(\mathbb{R}^n)\}$ is a $\sigma$-algebra.)
    \item[(b)] Prove that $\mathcal{B}(\mathbb{R}^n)$ is scaling invariant, i.e., if $A \subset \mathbb{R}^n$ is a Borel measurable set, then
    \[
    \lambda A = \{\lambda x \mid x \in A\}
    \]
    is a Borel measurable set for every $\lambda \in \mathbb{R}$.
\end{itemize}

\section{Hex and Such}
Let $A \subset [0,1]$ be the set of real numbers in $[0,1]$ having a hexadecimal expansion with the digit 5 appearing infinitely many times, and the ‘digit’ E appearing at most finitely many times. Prove that $A$ is a Borel set. (Hint: see p. 2 of Folland’s book.)

\section{Admissible Annuli}
Define an admissible annulus in $\mathbb{R}^2$ to be a set of the form
\[
\{(x, y) \in \mathbb{R}^2 \mid r^2 < (x - a)^2 + (y - b)^2 < R^2\},
\]
where $a, b \in \mathbb{Q}$, $r, R \in \mathbb{Q}_{>0}$, and $r < R$.

\begin{itemize}
    \item[(a)] Prove that there are only countably many admissible annuli.
    \item[(b)] Prove that every open subset of $\mathbb{R}^2$ is a countable union of (not necessarily disjoint) admissible annuli.
    \item[(c)] Prove that the Borel $\sigma$-algebra on $\mathbb{R}^2$ is generated by the collection of admissible annuli.
\end{itemize}

\section{Nur für Verrückte}
\subsection*{(It’s really not necessary to attempt these problems. Do not hand them in!)}

\begin{itemize}
    \item[(1)] Let $X$ be a set, and define two operations on $\mathcal{P}(X)$:
    \begin{itemize}
        \item The “product” of two subsets $E, F \subset X$ is the intersection $E \cap F$.
        \item The “sum” of two sets $E, F \subset X$ is the symmetric difference $E \Delta F$.
    \end{itemize}
    \begin{itemize}
        \item[(a)] Prove that these operations endow $\mathcal{P}(X)$ with the structure of a commutative ring. What are the additive and multiplicative units? Prove that this ring is idempotent.
        \item[(b)] Let us say that a nonempty subset $A \subset \mathcal{P}(X)$ is a ring if it is closed under differences and finite unions. In other words, if $E, F \in A$, then $E \setminus F \in A$ and $E \cup F \in A$. Prove that a subset $A \subset \mathcal{P}(X)$ is an algebra iff it is a ring containing $X$.
        \item[(c)] Prove that a nonempty subset $A \subset \mathcal{P}(X)$ is a ring iff it is a subring of $\mathcal{P}(X)$. Also prove that it is an algebra iff it is a subring containing the multiplicative identity.
    \end{itemize}
    \item[(2)] Let $(X, \mathcal{A})$ and $(Y, \mathcal{B})$ be measurable spaces. Say that a map $f : X \to Y$ is measurable (with respect to the $\sigma$-algebras $\mathcal{A}$ and $\mathcal{B}$) if $f^{-1}(E) \in \mathcal{A}$ for every $E \in \mathcal{B}$.
    \begin{itemize}
        \item[(a)] Prove that measurable spaces with measurable maps as morphisms form a category.
        \item[(b)] Try convincing an analyst that (a) is useful.
    \end{itemize}
\end{itemize}


\end{document}