\documentclass[11pt]{article}
\usepackage{amsmath,amssymb,alltt,fancyheadings, amsfonts}
%\usepackage{pdfsync}
\usepackage{times, mystyle}
\usepackage{tikz}
%\usepackage{xcolor}
\usepackage{enumitem}
\usepackage{hyperref}
   \usepackage{fancyvrb}
  
    
\usetikzlibrary{fixedpointarithmetic}


% \usepackage{tkz-euclide}
% \usetkzobj{all}

\newcommand*\Bell{\ensuremath{\boldsymbol\ell}}



\addtolength{\headheight}{3pt}

\addtolength{\textwidth}{100pt}
\addtolength{\evensidemargin}{-50pt}
\addtolength{\oddsidemargin}{-50pt}

\addtolength{\textheight}{120pt}
\addtolength{\topmargin}{-50pt}

\newcommand{\cA}{\mathcal{A}}
\newcommand{\cB}{\mathcal{B}}
\newcommand{\cP}{\mathcal{P}}
\newcommand{\ba}{\mathbf{a}}
\newcommand{\bi}{\mathbf{i}}
\newcommand{\bj}{\mathbf{j}}
\newcommand{\bk}{\mathbf{k}}
\newcommand{\br}{\mathbf{r}}
\newcommand{\bv}{\mathbf{v}}
\newcommand{\bF}{\mathbf{F}}
\renewcommand{\bS}{\mathbf{S}}
% \newcommand{\GL}{\protect\operatorname{GL}}
\newcommand{\proj}{\protect\operatorname{proj}}
\newcommand{\orth}{\protect\operatorname{orth}}
\newcommand{\curl}{\protect\operatorname{curl}}
\newcommand{\SP}{\protect\operatorname{Sp}}
%\newcommand{\R}{\protect\operatorname{R}}

\pagestyle{empty}
%\setlength{\headrulewidth}{0pt}
%\setlength{\footrulewidth}{0pt}

\VerbatimFootnotes 
\begin{document}


\thispagestyle{empty}


%%%%%%%%%%%%%%%%%%%%%%%%%%%%%%%%%%%%%%%%%%%%%%%%%%%%%%%%%%%%


\begin{center}
\textbf{%
Math 597, Winter 2025\\
Homework Set 0\\
}
\end{center}
\begin{center}
  \textit{Only some of the questions on this and other homework sets will be graded}.
\end{center}

  \begin{enumerate}
  \item[0.]
    \emph{Not graded}.
    Read Sections~0.1-0.3 and 0.5-0.6 in Folland's book. \textit{Note}: I expect you to have seen much but not necessarily all of this material in earlier courses. It is not necessary to know everything by heart right now. However, in order to succeed in the class, you need to be able to read mathematical material at this level of abstraction and (lack of) detail.

  \item[1.]
    (Approaching 597.)
    Let $A$ be an infinite (not necessarily countable) set, and
    $f\colon A\to\R$ a function. Supoose that for every integer $N\ge
    1$ there exist finite subsets $A^+_N\subset A$ and $A^-_N\subset A$ such that:
    \begin{itemize}
    \item[(i)]
      $|f(\alpha)|\le N^{-1}$ for all $\alpha\in A\setminus (A^+_N\cup A^-_N)$;
    \item[(ii)] 
      $\sum_{\alpha\in A^+_N}f(\alpha)\ge N$;
    \item[(iii)] 
      $\sum_{\alpha\in A^-_N}f(\alpha)\le-N$. 
    \end{itemize}
    Prove that for any $N\ge 1$ there exists a finite subset $B_N\subset A$ such that
    \[
      \left|597-\sum_{\alpha\in B_N}f(\alpha)\right|\le\frac1N.
    \]
    
  \item[2.]
    (Limsup and liminf.)
    Let $X$ be a nonempty set, and $A,B$ subsets of $X$. Define a sequence $(E_n)_{n=1}^\infty$ of subsets of $X$ by
    \[
      E_n=
      \begin{cases}
        A &\text{if $n$ is a prime number}\\
        B &\text{otherwise}\\
      \end{cases}
    \]
    Characterize the sets (see~\S0.1 in Folland for notation)
    \[
      \limsup E_n
      \quad\text{and}\quad
      \liminf E_n.
    \]
    
  \item[3.]
    (Polynomial convergence.)
    Let $f\colon\Z_{\ge0}\times\Z_{\ge0}\to\R$ be a function with the property that for every polynomial
    \[
      p(x)=x^d+a_1x^{d-1}+\dots+a_d
     \]
     with integer coefficients, we have that
     \[
       \lim_{n\to\infty}f(n,p(n))=\lim_{n\to\infty}f(p(n),n)=0.
     \]
     Does it follow that $f(m,n)\to 0$ as $m,n\to\infty$. In other words, given $\e>0$, does there exist $N\ge 0$ such that $|f(m,n)|<\e$ whenever $|m|,|n|\ge N$? Give a proof or a counterexample. 
    
  \end{enumerate}

\end{document}
